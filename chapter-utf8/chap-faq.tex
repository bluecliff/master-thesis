\chapter{总结与展望}
\label{chap:con}

\section{总结}
生物信息学的一个重要研究领域即序列分析,而序列分析的首要基础工作即短读比对(short reads alignment)。近年来也已经涌现出了很多成功
的短读比对程序,如bowtie,BWA,MAQ等在实践应用领域取得了很大的成功。我们在对压缩后缀数组的研究基础上,提出了基于压缩后缀数组的
短读序列比对算法,为序列分析提供了另一个可选的工具。

对于短读比对而言,需要考量的两个指标是比对时间和比对精度。比对时间由两个因素组成,一个是建立索引的时间,一个是比对时间。建立索引
采用压缩后缀数组由于内存耗用的限制,所以我们采用了增量法,时间效率有所下降,但空间效率得到提高,使得CSAA可以在普通PC上索引较大
规模的参考基因组。比对时间耗时大则是因为短读数量规模大,本文提出的解决方案是多线程处理短读序列。因为序列比对中天然的存在各个短读
比对过程相互独立的性质,可以非常高效的实现并行化,从而提高了比对的速度。除此之外,本文还应用了seed,分支预测等方法尽量减少无效的
搜索方向上的时间耗费,提高了CSAA的时间效率。

另一个需要考量的是比对精度问题。比对精度是考验一个比对算法的核心因素,本文提出了采用压缩后缀数组的后向搜索算法来做近似匹配的方
法,可以较为快速的完成短读的匹配,并且支持substitute,insert,delete三种操作,也即真正实现了对gap open 和gap extension 的全面支
持,因此CSAA可以实现相较bowtie更高的比对精度。

除了以上工作,本文实现的CSAA也对双端测序数据有较好的支持,双端测序为比对精度的提高提供了更多的依据。虽然双端比对消耗更多的时间,
但结合双端测序和distance数据,最终我们获取了更好的比对精度。

\section{进一步工作}
本文提出的CSAA虽然可以解决比对问题,但实际上还是有很多问题需要解决的。一个是建立索引的时间过大,CSA的构建时间太长,在构建本文使
用的人类基因组数据(2.8GB)的索引时,耗时26小时。如何更有效的构建索引是以后工作尚需解决的问题。另一个就是本文所能处理的测序数据类
型有限,只能处理Illumina测序得到的数据,对于最新的Solid测序数据尚还不能支持。最后就是CSAA比对程序在比对过程中并没有完全利用到测
序时核苷酸的质量分数数据,如何通过测序时核苷酸本身的质量分数来决定CSAA比对过程中的分支限界以及对近似序列的罚分,这是需要进一步
完成的工作,如能考虑这一因素,相信CSAA的比对精度会更高。最后就是如果CSAA在比对过程中把所有的"A,C,G,T"以外的所有核苷酸标记都转成
了"N",这是为了比对的方便。如果比对时考虑其本身的意义,对CSAA的精度的提高也是有好处的。

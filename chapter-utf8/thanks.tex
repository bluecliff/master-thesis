
\begin{thanks}

毕业论文暂告收尾,这也意味着我在西安电子科技大学的六年半的学习生活既将结束。
回首既往,自己一生最宝贵的时光能于这样的校园之中,能在众多学富五车、才华
横溢的老师们的熏陶下度过,实是荣幸之极。在这四年的时间里,我在学习上和思
想上都受益非浅。这除了自身努力外,与各位老师、同学和朋友的关心、支持和鼓
励都是分不开的。

论文的写作是枯燥而又充满挑战的,且生物信息学是交叉学科的典型,出身计算机
领域的我对生物学深感陌生,因而在整个毕业设计过程中都碰到了很多问题,所幸有
实验室各位老师的指导和教诲,实验室博士,硕士等学识渊博,经验丰富的同学的帮
助下,我顺利的完成了毕业设计和论文写作。在此,我特别要感谢我的导师霍红卫
教授。从论文的选题、文献的查找阅读、国内外最新成果的收集、实验设备的采购,
以及向同行专家老师的沟通请假等等方面,霍老师都提供了全面的帮助。没有霍老
师的帮助,我的论文是很难完成的。此外还有孙志刚博士提供的资料,陈龙刚提供的
压缩后缀数组索引程序等也为我的论文顺利完成提供了很大的帮助。同样的也要感谢
赵恒,聂宜静,赵玉豪,王哲,陈晓阳,赵睿醒等同学一起为实验室创造的优良科研
氛围,以及在学习上给我的帮助。

感谢所有关心、支持、和帮助我的的老师同学们。

时间的仓促及自身专业水平的不足,整篇论文肯定存在尚未发现的缺点和错误。
恳请阅读此篇论文的老师、同学,多予指正,不胜感激!
......

\vskip 18pt

谨把本文献给我最敬爱的父母亲以及所有关心我的人!

\end{thanks}
